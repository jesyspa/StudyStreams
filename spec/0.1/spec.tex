\documentclass[12pt,a4paper]{report}
\title{StudyStreams 0.1 Specification}
\author{Anton Golov}
\begin{document}
	\maketitle{}
	\tableofcontents{}
	\chapter{Introduction and Goals}
		StudyStreams is a C++ framework that intends to simplify learning
		the language by giving students the chance to write programs that
		are very similar to what a usual C++ program would look like, but
		would be automatically tested.  The \textbf{solution}, written by
		the \textbf{student}, uses \textbf{streams} to communicate with
		the \textbf{lesson} that is written by the \textbf{tutor}.  These
		streams use the \texttt{std::istream} and \texttt{std::ostream}
		interfaces, which allow them to behave like standard streams.

		A typical lesson consists of some sort of welcome, a number of
		exercises, and a parting message.  Most of the lesson structure is
		already in place, so that the teacher need only provide the input
		an exercise will give to the student, and the output it is to
		expect.
	\chapter{Usage Examples}
		\section{Alice}
		\section{Bob}
		\section{Charles}
	\chapter{Non-goals}
	\chapter{Lesson Interface}
	\chapter{Solution Interface}
\end{document}
