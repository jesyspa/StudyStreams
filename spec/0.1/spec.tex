\documentclass[12pt,a4paper]{report}
\title{StudyStreams 0.1 Specification}
\author{Anton Golov (\texttt{kdesevis@gmail.com})}
\begin{document}
	\maketitle{}
	\tableofcontents{}
	\chapter{Introduction and Goals}
		StudyStreams is a C++ framework that intends to simplify learning
		the language by giving students the chance to write programs that
		are very similar to what a usual C++ program would look like, but
		would be automatically tested.  The \textbf{solution}, written by
		the \textbf{student}, uses \textbf{streams} to communicate with
		the \textbf{lesson} that is written by the \textbf{tutor}.  These
		streams use the \texttt{std::istream} and \texttt{std::ostream}
		interfaces, which allow them to behave like standard streams.
		
		\begin{quote}
			\emph{Note:} Unless specified otherwise, all classes are
			described from the perspective of a student:  \texttt{OutStream}
			is a stream through which the solution passes the results of the
			program, not through which the lesson outputs questions.
		\end{quote}

		A typical lesson consists of some sort of welcome, a number of
		exercises, and a parting message.  Most of the lesson structure is
		already in place, so that the teacher need only provide the input
		an exercise will give to the student, and the output it is to
		expect.

		A typical solution is a C++ program written by the student which
		receives input from the lesson, processes it, and passes output
		back.  An incomplete solution can be presented by the teacher,
		which should give the student an idea of what is required.  A
		solution should ideally be possible to translate into standard
		C++ program by changing the included headers and using \texttt{std}
		instead of \texttt{study}.

		The focus of the first release is to be usable for the students.
		StudyStreams must be easy to install on Linux systems, and the
		lessons must be easy to compile.  By themselves, all lessons must
		compile successfully, but may not pass as they are.

		\section{Summary}
			To outline, the following are to be considered goals for version
			0.1 of StudyStreams, in no particular order.
			\begin{itemize}
				\item A config, build and install script for the static library.
				\item An implementation that is capable of running all currently
					available lessons.\footnote{Potentially running -- this just
					refers to the \texttt{study} library, not to a build system.}
				\item At least 15 lessons of various difficulty.
				\item A system for easy compilation of lessons.
				\item Operational lesson-input, lesson-output, and debug streams.
			\end{itemize}

		\begin{quote}
			\emph{Note:} In order to not repeat information, this spec does not
			document the classes available.  For details on those, see the
			\textit{Tutorial for Teachers} and \textit{Tutorial for Students}.
		\end{quote}

	\chapter{Usage Examples}
		In order to illustrate the nature of StudyStreams a little better,
		this section provides some situations in which they can come in handy.
		Suggestions for more examples are always welcome, please email them to
		the author of this spec.

		\section{Alice}
			Alice is an engineer in a small company producing intelligent
			toasters.  The last programmer has recently mysteriously
			disappeared, and now she has to quickly brush up her C++ in
			order to fix a critical bug that caused the products to grow
			legs.  Having no time to make herself a proper study schedule,
			she instead downloads StudyStreams and starts doing the exercises,
			while keeping a careful watch out for fiendish kitchen apparatus.

		\section{Bob}
			Ever since he became a professor in computer science, Bob has wished
			there'd be some easier way to grade the exams that the students had
			to hand in.  As things were, there was a series of input files and
			output files and pipes and Cthulhu knows what else, which sort of
			worked, but didn't allow the versality that Bob wanted.  Adding a new
			test was so much work, that he simply hasn't bothered the last five
			years, which the students had noticed and were using to its fullest.

			Deciding that such blatant cheating should not be allowed to continue,
			Bob downloaded StudyStreams and wrote up some custom lesson subclasses
			that represent the different type of tests.  Now, in order to add a new
			function, he simply has to derive from one of these subclasses and
			implement the desired function.

		\section{Charles}
			Charles is one of Bob's students, and while he is somewhat upset that
			he can no longer copy the work of older students, but otherwise doesn't
			mind the change:  a different namespace name for some objects, but
			otherwise no change.

	\chapter{Non-goals}
		At the moment, the following features are not expected, and no work
		should be done to implement them in the main branch, unless it proves
		trivial.  This list is likely to be updated as brilliant ideas are suggested.
		\begin{itemize}
			\item Tests based on anything other than streams.
			\item Automatic generation of lessons.
		\end{itemize}

